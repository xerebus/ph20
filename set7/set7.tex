%%% PREAMBLE

\documentclass{article}
\usepackage{amsmath,amssymb,amsthm,fullpage,graphicx,subcaption}

% Use break between paragraphs instead of indentation
%\usepackage{parskip}

% Code highlighting
\usepackage[utf8]{inputenc}
\usepackage{listings, color}

% Default fixed font does not support bold face
\DeclareFixedFont{\ttb}{T1}{txtt}{bx}{n}{9} % for bold
\DeclareFixedFont{\ttm}{T1}{txtt}{m}{n}{9}  % for normal

% Custom colors
\usepackage{color}
\definecolor{deepblue}{rgb}{0,0,0.5}
\definecolor{deepred}{rgb}{0.6,0,0}
\definecolor{deepgreen}{rgb}{0,0.5,0}

% Python style for highlighting
\newcommand\pythonstyle{\lstset{
language=Python,
basicstyle=\ttm,
otherkeywords={self},             % Add keywords here
keywordstyle=\ttb\color{deepblue},
emph={MyClass,__init__},          % Custom highlighting
emphstyle=\ttb\color{deepred},    % Custom highlighting style
stringstyle=\color{deepgreen},
frame=tb,                         % Any extra options here
showstringspaces=false,           % 
commentstyle=\ttm\color{magenta}
}}

% Python environment
\lstnewenvironment{python}[1][]
{
\pythonstyle
\lstset{#1}
}
{}

% Python for external files
\newcommand\pythonex[2][]{{
\pythonstyle
\lstinputlisting[#1]{#2}}}

% Python for inline
\newcommand\pyln[1]{{\pythonstyle\lstinline!#1!}}

% Various math conveniences
\theoremstyle{definition}
\newtheorem*{prop}{Proposition}
\newtheorem*{obs}{Observation}
\newtheorem*{lemma}{Lemma}
\newtheorem*{disc}{Discussion}
\newtheorem*{qn}{Question}
\renewcommand{\bar}{\overline}
\newcommand{\nin}{\not\in}
\renewcommand{\>}{\rangle}
\newcommand{\<}{\langle}



%%% FILE INFO

% Header
%\usepackage{fancyhdr}
%\usepackage[margin=1in, headheight=50pt]{geometry}
%\pagestyle{fancy}
%\lhead{\textbf{CS21}}
%\chead{Final}
%\rhead{Aritra Biswas}
%\setlength{\headsep}{20pt}

% Title stuff
\title{\textbf{Set 7: Cryptography}}
\date{}
\author{Aritra Biswas}



%%% DOCUMENT

\begin{document}

\maketitle

\section{Hypothetical Threat Model}

Suppose that I'm trying to collaborate with my friend Claire on writing a
website.
Claire is in Nauru, and for some reason, the Nauruan government has decided that
I might be a threat to the people of Nauru (all 10,000 of them).

Suppose that the Nauruan government has the ability and legal authority to
monitor Claire's network traffic, and the ability (but not the legal authority)
to interfere with it. They also do not have the authority to restrict
Claire's communication with me.

I don't want Claire to directly access our web server in the US because I would
then have to worry about protecting the web server. Instead, Claire and I resort
to exchanging code snippets via encrypted email.

\section{RSA}

The basic idea behind RSA is to create a cryptographic hash function such that:
\begin{enumerate}
\item the function is easy to apply with a public key,
\item the function is easy to invert with a private key, and
\item it is difficult to obtain the private key from the public key.
\end{enumerate}
In RSA, there is a modulus $n$, a public key $(n, e)$, and a private key $(n, d)$.
Requirement 3 is achieved because $d$ is the multiplicative inverse of $e$
modulus $\varphi(n)$; that is, $de = 1\text{ mod }\varphi(n)$.

Thus, calculating $d$
from $e$ requires knowing $\varphi(n)$, where $\varphi$ is Euler's totient function.
The modulus $n$ is a product of two prime numbers $p$ and $q$. If $p$ and $q$
are known, calculating $\varphi(n)$ is trivial: $\varphi(n) = \varphi(p)\varphi(q)
= (p-1)(q-1)$ since, for a prime number $k$, there are $k-1$ totatives
(numbers between $0$ and $k$, exclusive, that are coprime to $k$). However,
without already the prime factors $p$ and $q$, the only efficient way to
calculate $\varphi(n)$ is to factorize $n$ (i.e. obtain $p$ and $q$), which is
prohibitively difficult if $n$ is large.

The public key exponent $e$ is chosen to be a totative of $\varphi(n)$. The
creators of RSA used Fermat's little theorem to prove that, for any integer $m$,
$(m^e)^d = m\text{ (mod } n)$. Thus, we can express a message as an integer
$m$ using a publicly-known padding scheme, create an encrypted message
$c = m^e \text{ mod } n$, and then retrieve the original message by
$m = c^d \text{ mod } n$.

\subsection{Known Weaknesses}

Some known weaknesses of RSA include:
\begin{itemize}
\item \textbf{Insufficient randomness:} if the random number generation in a machine
that generates RSA keys is not well-seeded, it can generate two distinct keys
$n$ and $n'$ that have a common prime factor, i.e. $n=pq$ and $n' = pq'$.
Then computing the gcd of $n$ and $n'$ yields $p$, and both keys have
been factored and compromised. This is currently addressed by using good sources
of entropy (such as sensor noise or keystrokes) to generate good random numbers.
\item \textbf{Timing attacks:} the decryption process involves computing
$c^d\text{ mod } n$. If an attacker can measure the amount of time required to
decrypt certain values of $c$, he can predict $d$. This is remedied by
cryptographic blinding, in which a random number $r$ is chosen and the decryption
process computes $(r^ec)^d\text{ mod }n = rc^d\text{ mod }n$. Then, the decryption process
can multiply by the modular multiplicative inverse of $r$ to obtain the desired
$c^d\text{ mod }n$. This way, the computation time depends on $r$ as well as $d$ and $c$,
and if $r$ is chosen randomly, one cannot infer $d$ from the computation time.
\end{itemize}

\section{Toy Implementation of RSA}

\pythonex[firstline=1, lastline=84]{toy_rsa.py}

\subsection{Collision Attack}

The following function is a brute-force attempt at a
traditional collision attack (no chosen prefix),
where we aim to find two message $m_1 \neq m_2$ such that
$H(m_1)=H(m_2)$ where $H$ is our hash function.

\pythonex[firstline=86, lastline=98]{toy_rsa.py}

The brute-force method simply tries all possible messages sequentially
until it finds one with a hash that matches one that has previously
been generated.

After running this attack a few times, it seems to always return $(0, n)$.
Upon further investigation, it appears that $H(m_1) = H(m_1 + kn)$ for all
integers $k$. This means that while a collision can be found trivially,
it should be difficult to find an useful collision since the decoy
message $m_2$ must be related to the original message $m_1$ by
$m_2 = m_1 + kn$.

\section{Test Drive: Tor}

Tor is a network allowing anonymous communication through onion routing.
Suppose Alice and Bob are communicating, and Alice
is using Tor.

The Tor network consists of various nodes, or computers equipped to
relay information. Suppose Alice wishes to send a message to Bob. The message
must include Bob's address.
The Tor client obtains a list of known nodes and then picks, at random,
$n$ nodes in order.
The client takes the message (including Bob's address) and encrypts it
with the $n$th node's
public key. Then it adds the $n$th node's address
and encrypts the result with the $(n-1)$th node's public key,
and so on, finally adding the 2nd node's address and 
encrypting it with the 1st node's public key.

The encrypted ``onion'' is then sent to the first node, which decrypts the message
with its private key, uncovers the 2nd node's address, and sends it to the
2nd node. This chain
proceeds with the $i$th node decrypting the message, discovering the $(i+1)$th
node's address, and sending it to the $(i+1)$th node. Finally, the $n$th node
decrypts the message, finds Bob's address, and delivers it to Bob.

Suppose an attacker Mallory, who wants to eavesdrop on the conversation between
Alice and Bob, manages to accesses the $i$th node in this path (which is already unlikely
since the chain of nodes is randomly chosen). She can find out the address of the
$(i-1)$th node, and the $(i+1)$th node, but nothing else. To fully decrypt the
message and know both the source and destination addresses, Mallory would have to
access every Tor node involved in the chain between Alice and Bob.

\section{Proof of Work}

A screenshot of \pyln{systemd} log showing a running Tor service is attached.

\section{Password Security}

If an attacker is trying to guess a password, it is likely that he will
(1) use all available information - the user's name,
etc. and then (2) brute force methods.

To prevent strategy (1), it is important to not use any of the following in
a password:
\begin{itemize}
\item names and/or nicknames of the user or relatives
\item names of places, pets, etc.
\item anything that could be found by finding the user on social media
\end{itemize}

In a brute force method (2), the attacker is likely to first use a dictionary,
so it is important to not use full words. If a dictionary attack fails,
the attacker could brute-force individual characters. Suppose a password
can be at most $n$ letters long. Let $p$ be the number of characters in the
alphabet used. Then there are $p + p^2 + p^3 +  \cdots + p^n$ possible passwords.

The higher the number of possible passwords, the less likely a brute-force
approach will be to succeed. Thus, the user can increase $p$ (by using
special characters, etc.) or increase $n$ (use a longer password).
For a fixed $k$, $f(x) = k^x$ will eventually grow faster than $f(x)= x^k$,
so increasing $n$ is probably the more effective choice.

A strong password can be easily compromised if the attacker gains access to
tools that the user uses to remember the password (a piece of paper in a wallet
with the password written on it, password management software, etc.). Thus,
it is also important to strike a balance between password strength and
memorability, because a password that needs to be written down is inherently
unsafe.

\section{Conclusion and Application to Threat Model}

I had known before that randomness and prime numbers play a key role in
encryption; now I'm aware of the specific need to generate good random numbers
such that large keys do not have a common prime factor.

In reference to the threat model, suppose Claire and I each generate our
own public/private key pairs and encrypt our emails using them. It is important
for Claire to store these keys in a directory that cannot be accessed by any
program on her computer that can connect to the internet; otherwise, the Nauruan
government might be able obtain her key (if they access her computer).

The government is in a prime position to attempt a man in the middle attack
since they control Claire's internet access. Thus, it would be ideal for
Claire and I to have exchange keys before she left the country. If this is
not possible, it would be necessary to use some authentication mechanism
that would be difficult to duplicate. For example, we could agree to exchange
keys through voice recordings.

\end{document}
